\section{introduction}
FMCWレーダー搭載の自動運転車が増えるにつれ、他の自動車からの送信波による干渉が問題になってくる。
特に受信信号と干渉波の周波数差がFFT前のローパスフィルタのカットオフ周波数よりも小さい場合が問題で、不要な周波数成分がFFT出力から検出される。干渉周波数は時間領域ではパルスのように現れ、周波数領域では全ての周波数要素に広がるので、所望信号の周波数成分は干渉によって埋もれてしまう。 

\begin{itemize}
    \item [9-13] 干渉が発生する部分をゼロパディング →干渉区間が長いとターゲットの情報が失われる
    \item [21,22] 干渉信号の位相、振幅から干渉信号を再構成し元の信号から引く→ 実環境では位相ノイズがある[25]から位相情報を使うのは信頼性に欠ける
    \item [24] weighted-envelope normalization(AWEN)で干渉信号をセンシングし、その振幅を減衰させて干渉を抑制 → 経験則からパラメータを決定するからレーダーシステムの調整がいる 
\end{itemize}
シンプルかつ効果的な方法としてwavelet-denoising[26]を使う。通常wavelet denoisingでは信号を分解してノイズ、信号に対応する係数を閾値にかける[27]がローパスフィルタアウトプットには所望信号のsin波と干渉信号のパルス波が混在していて、干渉波は直接他のレーダーから飛んでくるから所望信号の電力より30dBくらい高い。[9]
ここでは所望信号をノイズに見立てて、強い電力をもつ干渉波をwaveletで推定→推定干渉波をミキサー出力から引いて、干渉を除去する方法を提案する。
        
        