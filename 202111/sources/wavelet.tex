\section{proposed mutual interference suppresion using wavelet denoising}
ミキサー出力をローパスフィルタにかけた出力は、干渉波と送信信号の傾きが異なる場合(Fig3(a)みたいな時)
\begin{equation}
    L_{I}(t)=L(M(t))+L\left(S(t) I^{D S}(t)\right)
\end{equation}
となる。この論文では干渉成分$L\left(S(t) I^{D S}(t)\right)$の電力が所望信号$L(M(t))$に比べて非常に大きいので、まず$L\left(S(t) I^{D S}(t)\right)$をwavelet変換を使って推定する。その後、推定したパラメータを使って$L\left(S(t) I^{D S}(t)\right)$を再構成して$L_I(t)$から取り除くことで所望信号$L(M(t))$を得る。
\begin{equation}
    \hat{L}(M(t))=L_{I}(t)-\hat{L}\left(S(t) I^{D S}(t)\right)
\end{equation}

\subsection{Decomposition of Low-Pass Filter Output Using Wavelet Transform}
Haarウェーブレットのマザーウェーブレット関数$\psi(t)$は
\begin{equation}
    \psi(t)=\left\{\begin{array}{lc}
    1 & (0 \leq t<1 / 2) \\
    -1 & (1 / 2 \leq t<1) \\
    0 & (t<0, t \geq 1)
    \end{array}\right.
\end{equation}

$L_I(t)$への離散ウェーブレット変換は
\begin{equation}
    W_{a, b}=\int_{-\infty}^{\infty} L_{I}(t) \psi_{a, b}^{*}(t) d t
\end{equation}
で表され、
\begin{align}
    \psi_{a, b}(t) &= 2^{\frac{a}{2}} \psi\left(2^{a} t-b\right)\cr
                   &= 2^{\frac{a}{2}} \psi\left(\frac{t-2^{-a}b}{2^{-a}}\right) \quad\left(a=1,2, \cdots, a_{T}\right) .
\end{align}
である。$a,b$はそれぞれスケーリング要素と時間要素になっていて、$a$を大きくすれば$\psi_{a,b}$は収縮する。
\begin{figure}[H]
    \centering
    \includegraphics[width=\linewidth]{Components/wavelet_example.jpg}    
\end{figure}

前についてる$2^{\frac{a}{2}}$は正規化係数で各スケールでの基底の電力の合計が$1$になるように正規化されている。

\subsection{threholding}
ウェーブレット変換で係数$W_{a,b}$が得られたら、どの係数が干渉信号に対応するか閾値から判定する。
一般的によく使われるのはsoft threholdingとhard thresholdingでどちらも$\lambda$より絶対値が小さい$W_{a,b}$は消される。
\begin{align}
    f_s(W_{a,b}) &= 
        \begin{cases}
            W_{a,b}-sgn(W_{a,b})\lambda & |W_{a,b}|\geq \lambda \\
            0, & otherwise
        \end{cases} \cr
    f_h(W_{a,b}) &= 
    \begin{cases}
        W_{a,b}, & |W_{a,b}|\geq \lambda \\
        0, & otherwise
    \end{cases} 
\end{align}

ここでは[33]で与えられるmodified-universal thresholdを使っている。
\begin{align}
    \lambda = \hat{\sigma}_a \sqrt{2\log(N_I)}
\end{align}